\documentclass[12pt]{article}

\usepackage{xeCJK}
\usepackage{ctex}
\usepackage{pdfpages}
\usepackage{titlesec}
\usepackage{fontspec}
\usepackage{booktabs}
\usepackage{graphicx}
\usepackage{float}
\usepackage{subfigure}
\usepackage{listings}
\usepackage{amsmath}
\usepackage{amssymb}
\usepackage{geometry}
\usepackage{xcolor}
\usepackage{fancyhdr}
\usepackage[hidelinks]{hyperref}
\usepackage{algpseudocode}
\usepackage{algorithm}
\usepackage{bm}
\usepackage{threeparttable}

\pagestyle{fancy}
\fancyhf{}
\renewcommand{\headrulewidth}{0pt}
\fancyfoot[L]{程序设计实践报告}
\fancyfoot[R]{\thepage}
\fancyfoot[C]{}

\geometry{a4paper,footskip=4cm}

\titleformat{\title}
{\centering\bfseries \zihao{4} \heiti}
{\thesection}
{1em}
{}

\titleformat{\section}
{\centering \color{xblue} \bfseries \zihao{4} \heiti}
{\thesection}
{1em}
{}

\titleformat{\subsection}
{\raggedright\bfseries \zihao{-4} \heiti}
{\thesubsection}
{1em}
{}

\titleformat{\subsubsection}
{\raggedright\bfseries \zihao{-4} \heiti}
{\thesubsubsection}
{1em}
{}

\definecolor{xblue}{RGB}{0,0,148}

% \setCJKmainfont{simsun.ttc}

\setmonofont{CascadiaCode.ttf}

\lstset{
    backgroundcolor=\color{white},
    keywordstyle=\color{blue},
    numberstyle=\color[RGB]{0,192,192},
    commentstyle=\color[RGB]{0,96,96},
    stringstyle=\color[RGB]{128,0,0},
    columns=fullflexible,
    breaklines=true,
    captionpos=b,
    tabsize=4,
    numbers=left,
    frame=single,
    basicstyle=\ttfamily
}

\urlstyle{same}

\floatname{algorithm}{算法}
\renewcommand{\algorithmicrequire}{\textbf{输入:}}
\renewcommand{\algorithmicensure}{\textbf{输出:}}

\begin{document}

\begin{titlepage}

\newgeometry{top=2cm,left=2cm}

\parbox[c]{8cm}{
    \includegraphics[width=8cm]{assets/hdu.png}
}

\setlength{\parindent}{0pt}
\centering
\vfill
{\zihao{-0} \heiti \textcolor{xblue}{程序设计实践报告}\par}
\vspace{10pt}
{\zihao{-2} \heiti 基于 C 语言的排序算法设计以及效率分析\par}
\vfill
{\large \heiti 卓越学院\par
计算机科学英才班\par
鲍溶\par
23060827\par
}
\vfill
\restoregeometry

\end{titlepage}

\renewcommand\contentsname{\textcolor{xblue}{目录}}
    \tableofcontents
\clearpage
\setcounter{page}{1}

\section{问题描述}

编写程序, 实现冒泡排序、 简单选择排序、 简单插入排序、 归并排序、 快速排序函数、 调用标准库函数排序、 其它特殊排序算法 (至少一种), 并产生规模分别为 1000、 10000、 100000、 1000000、 10000000 的模拟数组, 使用上述排序方法分别对同样的模拟数据进行排序, 在验证排序结果正确性的同时, 利用系统时间函数分别记录各排序开始前时间和排序完成时时间, 计算出各排序所需时间. 再对已排序数据稍加次序调整, 模拟几乎有序数组, 再重复上述排序过程.

总结、 分析程序测试结果, 撰写排序算法实现与性能评测专题报告, 报告内容完整、 规范, 内容需包括: 封面、 目录、 问题描述、 主要算法 (如果有: 和数据结构设计)、 测试结果、 结果分析、 小结, 附规范完整程序.

需注意, 各排序算法需要通过函数参数与外部待排序序列交换信息, 函数内不使用全局变量, 体现出模块化程序设计原则, 便于算法推广使用; 各排序结果需使用检测函数检测正确性; 测试结果记录在表格里, 纵向是问题规模、 横向是排序方法, 这样可以提供直观对比效果, 分无序序列和几乎有序序列两个评测结果表格, 超时需说明标准; 测试结果分析应该包含算法理论时间复杂度和测试结果是否吻合分析.

\section{主要算法}

\subsection{冒泡排序}

算法 \ref{algo_bubblesort} 描述了冒泡排序算法. 该算法持续交换不满足给定关系的元素以达到排序效果.

\begin{algorithm}[H]
\caption{冒泡排序.}
\label{algo_bubblesort}
\begin{algorithmic}[1]
\Require 待排序数组 $\bm{a},$ 全序关系 $\preceq,$ 待排序数组长度 $n.$
\Ensure 排序完成的数组.
\For{$i \gets 1 \hdots n$}
    \For{$j \gets 1 \hdots n - i$}
        \If{$\bm{a}[j] \npreceq \bm{a}[j + 1]$}
            \State $\mathsf{Swap}(\bm{a}[j], \bm{a}[j + 1])$
        \EndIf
    \EndFor
\EndFor
\State \Return $\bm{a}$
\end{algorithmic}
\end{algorithm}

\subsection{直接插入排序}

算法 \ref{algo_straight_insertion} 描述了直接插入排序. 该算法通过将各元素插入至已排序区间中的正确位置从而实现排序.

\begin{algorithm}[H]
\caption{直接插入排序.}
\label{algo_straight_insertion}
\begin{algorithmic}[1]
\Require 待排序数组 $\bm{a},$ 全序关系 $\preceq,$ 待排序数组长度 $n.$
\Ensure 排序完成的数组.
\For{$i \gets 2 \hdots n$}
    \State $p \gets \bm{a}[i]$
    \State $j \gets i - 1$
    \While{$j \ge 0 \land \bm{a}[j] \npreceq p$}
        \State $\bm{a}[j + 1] \gets \bm{a}[j]$
        \State $j \gets j - 1$
    \EndWhile
    \State $\bm{a}[j + 1] \gets p$
\EndFor
\State \Return $\bm{a}$
\end{algorithmic}
\end{algorithm}

\subsection{直接选择排序}

由算法 \ref{algo_straight_selection} 描述的直接选择排序的每一趟都选择未排序元素中给定序下的最小元素, 并将其放置于已排序元素区间的末尾.

\begin{algorithm}[H]
\caption{直接选择排序.}
\label{algo_straight_selection}
\begin{algorithmic}[1]
\Require 待排序数组 $\bm{a},$ 全序关系 $\preceq,$ 待排序数组长度 $n.$
\Ensure 排序完成的数组.
\For{$i \gets 1 \hdots n$}
    \State $p \gets \mathsf{IndexMin}(\bm{a}[i+1 \dots n], \preceq)$
    \State $\mathsf{Swap}(\bm{a}[p], \bm{a}[i])$
\EndFor
\State \Return $\bm{a}$
\end{algorithmic}
\end{algorithm}

\subsection{归并排序}

由算法 \ref{algo_merge} 描述的归并排序采用分治策略, 将每个有序子数组按序合并得到已排序的最终结果. 若将两个有序子数组记为 $\bm{s}_1$ 和 $\bm{s}_2,$ 对这两个数组的合并仅需要执行 $\min \left\{\textsf{Len}(\bm{s}_1), \textsf{Len}(\bm{s}_2)\right\} - 1$ 次比较.

\begin{algorithm}[H]
\caption{归并排序.}
\label{algo_merge}
\begin{algorithmic}[1]
\Require 待排序数组 $\bm{a},$ 全序关系 $\preceq,$ 待排序数组长度 $n.$
\Ensure 排序完成的数组.
\If{$\mathsf{Len}(\bm{a}) = 1$}
    \State \Return $\bm{a}$
\EndIf
\State $p \gets \mathsf{Len}(\bm{a}) / 2$
\State $\bm{l} \gets$ 以 $\bm{a}[1 \dots p]$ 为参数递归调用自身
\State $\bm{h} \gets$ 以 $\bm{a}[p+1 \dots n]$ 为参数递归调用自身
\State $\bm{a} \gets \mathsf{MergeOrdered}(\bm{l}, \bm{h})$
\State \Return $\bm{a}$
\end{algorithmic}
\end{algorithm}

\subsection{以最末位元素为主元的快速排序}

由算法 \ref{algo_quicksort_highpivot} 描述的快速排序算法同样使用了分治的思想. 给定待排序数组 $\bm{a},$ 该算法选取一主元 $p,$ 确定其在排序后数组中的正确位置, 同时将剩余元素分成两个集合 $\mathbb{S}_l$ 和 $\mathbb{S}_h,$ 满足:

\begin{equation}[H]
  \left\{
    \begin{aligned}
        \mathbb{S}_l & = \left\{x \in \bm{a}\ \vert\ x \preceq p \right\} \\
        \mathbb{S}_h & = \left\{x \in \bm{a}\ \vert\ x \npreceq p \right\}
    \end{aligned}
  \right.
\end{equation}

之后, 再在 $\mathbb{S}_l$ 和 $\mathbb{S}_h$ 上分别递归调用自身, 直到输入数组中没有元素或只有一个主元.

算法 \ref{algo_quicksort_highpivot} 中采用的主元选取策略是取数组最末元素为主元. 其他策略也是可行的.

\begin{algorithm}[H]
\caption{末主元快速排序.}
\label{algo_quicksort_highpivot}
\begin{algorithmic}[1]
\Require 待排序数组 $\bm{a},$ 全序关系 $\preceq,$ 待排序数组长度 $n.$
\Ensure 排序完成的数组.
\If{$\mathsf{Len}(\bm{a}) = 1$}
    \State \Return $\bm{a}$
\EndIf
\State $p \gets \bm{a}[n]$
\State $i \gets 1$
\For{$j \gets 1 \dots n$}
    \If{$\bm{a}[j] \preceq p$}
        \State $\mathsf{Swap}(\bm{a}[i], \bm{a}[j])$
        \State $i \gets i + 1$
    \EndIf
\EndFor
\State $\mathsf{Swap}(\bm{a}[i], \bm{a}[n])$
\State $\bm{l} \gets$ 以 $\bm{a}[1 \dots i-1]$ 为参数递归调用自身
\State $\bm{h} \gets$ 以 $\bm{a}[i+1 \dots n]$ 为参数递归调用自身
\State $\bm{a} \gets \mathsf{Concat}(\bm{l}, [p], \bm{h})$
\State \Return $\bm{a}$
\end{algorithmic}
\end{algorithm}

\subsection{由 \texttt{stdlib.h} 提供的 \texttt{qsort}}

该算法通过调用 \texttt{stdlib.h} 提供的 \texttt{qsort} 函数实现. C 语言标准中并未指定该函数所使用的算法, 同时也未对其性能作任何保证. 因此, 在本次实验中, 我们仅仅假定该函数已被正确实现, 其包装如算法 \ref{algo_stdlib_qsort} 所示.

\begin{algorithm}[H]
\caption{由 \texttt{stdlib.h} 提供的 \texttt{qsort} 排序.}
\label{algo_stdlib_qsort}
\begin{algorithmic}[1]
\Require 待排序数组 $\bm{a},$ 全序关系 $\preceq,$ 待排序数组长度 $n.$
\Ensure 排序完成的数组.
\State $\bm{a} \gets \mathtt{qsort}(\bm{a}, \preceq, n)$
\State \Return $\bm{a}$
\end{algorithmic}
\end{algorithm}

\subsection{基于 QuickPerm 的确定性 Bogosort}

由算法 \ref{algo_bogosort_determ} 描述的确定性 Bogosort 作为低效率排序算法的典型进入本次实验的讨论范围. 该算法通过持续对输入数组求取下一个全排列的方法寻找一个满足给定序关系的排列. 原始的 Bogosort 使用一 Las Vegas 型随机算法生成全排列, 具有运行时间难以确定的劣势. 因此, 我们采用 QuickPerm\cite{bib_quickperm} 这一确定性算法实现排列的求取. 我们可以由此得出推论, 该算法是一必然能够终止的确定性算法.

\begin{algorithm}[H]
\caption{确定性 Bogosort.}
\label{algo_bogosort_determ}
\begin{algorithmic}[1]
\Require 待排序数组 $\bm{a},$ 全序关系 $\preceq,$ 待排序数组长度 $n.$
\Ensure 排序完成的数组.
\For{$\bm{p} \in \mathsf{QuickPerm}(\bm{a})$}
    \For{$i \gets 2 \dots n$}
        \If{$\bm{p}[i - 1] \npreceq \bm{p}[i]$}
            \State 检验下一个排列 $\bm{p}$
        \EndIf
    \EndFor
    \State \Return $\bm{p}$
\EndFor
\end{algorithmic}
\end{algorithm}

\subsection{扩展基数排序}

如算法 \ref{algo_radix} 描述的扩展基数排序是基于分布而不是比较排序的一种. 它以每一 "数位" 为单元, 由最低数位到最高数位依次进行计数排序. 给定一满足数位权重递增规律的数位提取函数 $\mathsf{Digit}(\cdot, \cdot),$ 以及一数位计数函数 $\mathsf{NDigits(\cdot)},$ 则定义于其上的扩展基数排序具有确定行为.

\begin{algorithm}[H]
\caption{扩展基数排序.}
\label{algo_radix}
\begin{algorithmic}[1]
\Require 待排序数组 $\bm{a},$ 数位提取函数 $\mathsf{Digit}(\cdot, \cdot),$ 数位计数函数 $\mathsf{NDigits(\cdot)},$ 待排序数组长度 $n.$
\Ensure 排序完成的数组.
\State $m \gets \mathsf{NDigits}(\bm{a}\ \text{中元素的类型})$
\State $\bm{b} \gets$ 长度为 $n$ 的数组
\For{$i \gets 1 \dots m$}
    \State $k \gets \mathrm{Card}\left(\mathrm{R}(\mathsf{Digit}(\bm{a}\ \text{中元素的类型}, i))\right)$
    \State $\bm{d} \gets$ 长度为 $k$ 的全零数组
    \For{$j \gets 1 \dots n$}
        \State $t \gets \mathsf{Digit}(\bm{a}[j], i)$
        \State $\bm{d}[j] \gets \bm{d}[j] + 1$
    \EndFor
    \For{$j \gets 2 \dots k$}
        \State $\bm{d}[j] \gets \bm{d}[j] + \bm{d}[j - 1]$
    \EndFor
    \For{$j \gets n \dots 1$}
        \State $t \gets \mathsf{Digit}(\bm{a}[j], i)$
        \State $\bm{b}[\bm{d}[t]] \gets \bm{a}[j]$
        \State $\bm{d}[t] \gets \bm{d}[t] - 1$
    \EndFor
    \State $\bm{a} \gets \bm{b}$
\EndFor
\State \Return $\bm{a}$
\end{algorithmic}
\end{algorithm}

\section{实验设计}

在本次实验中, 我们需要在验证所实现的排序算法的正确性的情况下, 比较不同排序算法在不同特征输入下的性能.

\subsection{假设与简化}

本实验中所实现的算法均为形如 $\mathsf{Sort}(\cdot, \cdot)$ 的函数. 每个算法接受两个参数: 任意数据类型的数组 $\bm{a}$ 和全序关系 $\preceq.$ 这两个参数决定了算法的行为.

在使用 C 语言实现算法时, 数组 $\bm{a}$ 由三元组 $(a, s, c)$ 唯一确定, 其中 $a$ 为该数组的首地址, 类型为 \texttt{void *}; $s$ 为数组中每元素所占据的字节数, 类型为 \texttt{size\_t}; $c$ 为数组中元素的个数, 类型为 \texttt{size\_t}. 全序关系 $\preceq$ 由函数指针实现, 类型为 \texttt{int (*)(const void *, const void *)}. 排序函数与全序关系的均与标准库兼容, 具体实现细节见附录.

本实验中, 编译系统采用 x64 平台上的 MSVC 19.38.33133, 测试程序运行于 Intel Core i9-13900HX 处理器, 架构为 Raptor Lake. 为了模拟真实的生产环境, 编译时开启 \texttt{/O2 /Oi /GL /arch:AVX2 /std:c17 /permissive-} 优化选项, 并将排序函数的实现以静态库的形式提前编译, 静态链接至测试驱动程序中, 以防止编译器通过对输入数据流的分析采取过多优化, 使得实验结果无法推广. 在编写 $\mathsf{Swap}(\cdot, \cdot)$ 和 $\preceq$ 等热点函数时, 采取无分支与手动小数据特化等措施, 以降低这些函数带来的额外开销.

\subsection{正确性检验}

由于标准库提供了 \texttt{qsort} 函数作为参考实现, 我们可以对比自行实现排序算法的输出结果与标注库 \texttt{qsort} 的输出结果, 以对算法正确性进行不完全归纳证明. 这里, 我们设计算法 \ref{algo_correctness} 进行正确性检验.

\begin{algorithm}[H]
\caption{使用不完全归纳检验排序算法的正确性.}
\label{algo_correctness}
\begin{algorithmic}[1]
\Require 待检测排序函数 $\mathsf{Sort}(\cdot, \cdot),$ 参考排序函数 $\mathsf{Sort}^*(\cdot, \cdot),$ 待排序数组长度 $n,$ 全序关系 $\preceq.$
\Ensure 检验的结果.
\State $\bm{a}_1 \gets$ 新生成的 $n$ 元待排序数组
\State $\bm{a}_2 \gets \bm{a}_1$
\State $\bm{a}_1 \gets \mathsf{Sort}(\bm{a}_1, \preceq)$
\State $\bm{a}_2 \gets \mathsf{Sort}^*(\bm{a}_2, \preceq)$
\For{$i \gets 1 \hdots n$}
    \If{$\bm{a}_1[i] \neq \bm{a}_2[i]$}
       \State \Return 检验失败
    \EndIf
\EndFor
\State \Return 检验通过
\end{algorithmic}
\end{algorithm}

\subsection{输入生成}

本次实验要求测量算法在不同特征输入下的性能情况, 包括随机分布输入与几乎有序输入. 在给定任意确定性均匀分布随机数生成算法 $\mathsf{Rand}$ 的情况下, 我们能够使用算法 \ref{algo_gen_rand} 产生所需长度的随机数组.

\begin{algorithm}[H]
\caption{生成给定长度的随机分布输入数组.}
\label{algo_gen_rand}
\begin{algorithmic}[1]
\Require 确定性均匀分布随机数发生算法 $\mathsf{Rand},$ 期望输入长度 $n.$
\Ensure 新生成的 $n$ 元随机数组.
\State 初始化 $\mathsf{Rand}$
\State $\bm{a} \gets []$
\For{$i \gets 1 \hdots n$}
    \State $\bm{a}[i] \gets$ 使用 $\mathsf{Rand}$ 生成的下一个随机数
\EndFor
\State \Return $\bm{a}$
\end{algorithmic}
\end{algorithm}

实验中, 我们采用梅森旋转算法 \cite{bib_mersenne_randgen} 产生均匀且具有极长周期的随机 32 位整数.

在获取了一个均匀的随机数组后, 我们可以借助参考排序算法 $\mathsf{Sort}^*(\cdot)$ 得到一个有序的相同维数的数组. 在该数组基础上对少量元素进行随机互换, 即可得到符合 "几乎有序" 条件的输入数组. 该两步过程由算法 \ref{algo_gen_rising} 和 \ref{algo_gen_almost_rising} 描述.

\begin{algorithm}[H]
\caption{生成给定长度的有序输入数组.}
\label{algo_gen_rising}
\begin{algorithmic}[1]
\Require 均匀随机数组 $\bm{a},$ 参考排序算法 $\mathsf{Sort}^*(\cdot, \cdot),$ 全序关系 $\preceq.$
\Ensure 新生成的有序数组.
\State $\bm{a} \gets \mathsf{Sort}^*(\bm{a}, \preceq)$
\State \Return $\bm{a}$
\end{algorithmic}
\end{algorithm}
\begin{algorithm}
\caption{生成给定长度的几乎有序输入数组.}
\label{algo_gen_almost_rising}
\begin{algorithmic}[1]
\Require 确定性均匀分布随机数发生算法 $\mathsf{Rand},$ 有序数组 $\bm{a},$ 交换元素对数 $p,$ 映射 $f: \mathrm{R}(\mathsf{Rand}) \to [1 \dots n].$
\Ensure 新生成的几乎有序数组.
\State 初始化 $\mathsf{Rand}$
\For{$i \gets 1 \hdots p$}
    \State $l \gets f(\text{使用}\ \mathsf{Rand}\ \text{生成的下一个随机数})$
    \State $h \gets f(\text{使用}\ \mathsf{Rand}\ \text{生成的下一个随机数})$
    \State 交换 $\bm{a}[l]$ 与 $\bm{a}[h]$
\EndFor
\State \Return $\bm{a}$
\end{algorithmic}
\end{algorithm}

实验中我们将 $p$ 固定为 16.

同时需要注意的是, 尽管在实现各排序算法时, 我们采用了针对任意大小元素的通用接口, 但是在本次实验中我们仅以 32 位无符号整形作为唯一的元素大小, 元素尺寸对排序性能的影响有待进一步研究.

\subsection{性能测量}

在操作系统无关的层面上, C 语言并没有为监视程序的运行提供便利的接口. 若仅仅使用每个排序算法在给定输入下的总运行时间作为性能的唯一判据, 虽然在理论上是相当可靠的, 但是鉴于部分排序算法在输入规模扩大时不具备在可接受时间内完成计算的可能, 我们必须寻找另一种能够在算法运行中实现提前停止的机制.

线程和进程是在运行期开启新控制流并对其进行监视的较好方法, 但是 C 语言的线程标准库并未得到广泛的支持 \cite{bib_msvc_threads}, 因此我们放弃了该方案, 转而寻求侵入式的性能测量方案. 由于元素之间的比较次数是排序算法重要的性能指标, 我们以统计元素比较次数的方式实现第二种性能统计方式. 同时, 当元素比较次数超过一预设阈值时, 我们便使被监视排序算法提前终止并报告错误, 以实现时间效益上的提升.

由于标准库提供的 \texttt{qsort} 并未暴露内部实现, 同时 C 语言在函数类型操作方面的表达能力较弱, 我们无法测量比较函数被 \texttt{qsort} 调用的次数并提前终止该算法. 因此, 我们将仅测量该算法的运行时间.

\section{结果分析}

每种算法均在随机分布, 有序和几乎有序三种输入分布和 1e3, 1e4, 1e5, 1e6 和 1e7 的输入规模下进行三轮测试, 取耗时和比较次数的平均值为最终结果. 测量结果见表 \ref{table_result_random} 和表 \ref{table_result_almost_rising}.

\begin{table}
\centering
\begin{threeparttable}
    \caption{不同规模的随机分布输入下排序算法的平均耗时统计}
    \begin{tabular}{cccccc}
        \toprule
         & \multicolumn{5}{c}{输入规模} \\
        排序算法 & 1e3 & 1e4 & 1e5 & 1e6 & 1e7 \\
        \midrule
        冒泡 & 0.001141 & 2.702836 & 13.730024* & 6.077534* & 7.011685* \\
        直接插入 & 0.000652 & 1.036735 & 13.070797* & 13.070870* & 11.028969* \\
        归并 & 0.000096 & 0.002270 & 0.026943 & 3.013913 & 8.062880 \\
        末主元快排 & 0.000099 & 0.001294 & 0.015638 & 1.090419 & 3.094441 \\
        直接选择 & 0.001589 & 2.042786 & 7.027422* & 7.048163* & 7.075249* \\
        标准库 \texttt{qsort} & 0.000051 & 0.001383 & 0.014345 & 1.067867 & 10.018054 \\
        确定性 Bogosort & 8.739827* & 11.071495* & 11.709166* & 13.703622* & 11.710889* \\
        扩展基数 & 0.000047 & 0.000552 & 0.005745 & 0.021736 & 2.028460 \\
        \bottomrule
    \end{tabular}
    \begin{tablenotes}
        \small
        \item 单位: 秒, 舍入到第 6 位小数.
        \item[*] 全部 3 次运行中比较次数均超过 $0\mathrm{x}40000000 = 1073741824$ 次的限制, 触发提前停止.
    \end{tablenotes}
    \label{table_result_random}
\end{threeparttable}
\end{table}

\begin{table}
\centering
\begin{threeparttable}
    \caption{不同规模的几乎有序输入下排序算法的平均耗时统计}
    \begin{tabular}{cccccc}
        \toprule
         & \multicolumn{5}{c}{输入规模} \\
        排序算法 & 1e3 & 1e4 & 1e5 & 1e6 & 1e7 \\
        \midrule
        冒泡 & 0.000832 & 1.086718 & 12.082078* & 6.707536* & 5.036540* \\
        直接插入 & 0.000026 & 0.000666 & 0.006648 & 0.067337 & 5.374296 \\
        归并 & 0.000060 & 0.001588 & 0.017266 & 1.096024 & 3.699776 \\
        末主元快排 & 0.000269 & 0.055786 & 6.740724\textsuperscript{!} & 7.710644* & 8.381640* \\
        直接选择 & 0.001601 & 2.042761 & 7.019613* & 7.042990* & 7.395880* \\
        标准库 \texttt{qsort} & 0.000016 & 0.000451 & 0.014345 & 0.070739 & 7.051840 \\
        确定性 Bogosort & 11.397016* & 10.738460* & 11.379927* & 11.389746* & 11.695746* \\
        扩展基数 & 0.000014 & 0.000526 & 0.005214 & 0.021780 & 2.037472 \\
        \bottomrule
    \end{tabular}
    \begin{tablenotes}
        \small
        \item 单位: 秒, 舍入到第 6 位小数.
        \item[*] 全部 3 次运行中比较次数均超过 $0\mathrm{x}40000000 = 1073741824$ 次的限制, 触发提前停止.
        \item[!] 3 次运行中部分趟比较次数超过 $0\mathrm{x}40000000 = 1073741824$ 次的限制, 触发提前停止.
    \end{tablenotes}
    \label{table_result_almost_rising}
\end{threeparttable}
\end{table}

\subsection{基于选择与交换的排序性能}

在两种输入分布下, 标准库 \texttt{qsort} 均能保持基线水准的性能表现. 这主要归因于 MSVC 平台使用 Introsort 算法实现该函数, 在快速排序基础上尝试降低其最坏情况下的比较次数. 与手动实现的快速排序相比较, 其在几乎有序输入上也能够具有较为理想的性能, 且运行时间相对于输入规模的变化趋势与输入分布相关性较弱.

直接插入排序在随机序列上表现不佳, 其需要进行 $\mathcal{O}(n^2)$ 次比较操作. 但是在几乎有序序列上表现尚可, 因为其可以滑过大部分已排序区间而不进行交换和重复的比较. 该算法的最差情况出现于完全逆序输入, 此时其需要进行 $(n^2 - n) / 2$ 次比较.

直接选择排序需要两两比较所有元素, 在每一趟中寻找未排序区间中给定序下的最小值. 因此, 无论输入分布如何, 均需要进行 $n^2$ 次比较操作.

冒泡排序在每一趟中交换不满足序关系的相邻元素. 因此, 如果输入已完全有序, 则在第一趟中便无需进行任何交换, 仅需要 $n - 1$ 次比较即可确认排序完成. 但是在完全逆序输入下, 它需要对所有逆序对进行比较并交换, 次数为 $(p^2 - p) / 2,$ 其中 $p$ 为元素对数.

\subsection{基于合并的排序性能}

归并排序是递归算法, 需要与输入等长的辅助空间. 归并排序在输入规模为 $n$ 时, 第 $i$ 层深度的调用仅需要 $n / 2^i$ 次比较, 因此该算法的总比较次数为 $n \log_2 n.$ 在数据量大且不考虑额外空间消耗的情况下, 该算法十分有效.

\subsection{基于全排列的排序性能}

基于 QuickPerm 的 Bogosort 求取给定数组元素的全排列直到完全有序. 显然, 最坏情况下需要进行 $n!(n-1)$ 次比较才能确定其已完全排列. 显然, 该算法缺少实际应用价值, 仅仅作为一种有趣的慢速排序算法在这里进行讨论.

\subsection{基于分布的排序性能}

扩展基数排序排序在所有输入规模下性能良好, 这主要归因于它的运行时间随输入增多而线性增加. 在扩展基数排序的前提有意义时, 它是一种高效的排序算法.

\section{小结}

本文对多种排序算法在不同的输入规模与输入分布下进行了性能比较, 分析了不同算法在比较次数与运行时间上的优劣, 得出需要针对输入数据的特征进行针对性算法选择的结论. 同时, 也可以采用多种启发式方法动态混合使用不同的排序算法, 以达到降低平均运行时间的效果.

本文没有针对某种特定的处理器和内存体系结构进行平台相关的分析研究. 在某些平台上可能存在更高效的常数优化方法能够降低算法运行的时间, 但是本文所分析得到的趋势与结论依然成立.

\begin{thebibliography}{}

\bibitem{bib_quickperm} Permutation Algorithms Using Iteration and the Base-N-Odometer Model (Without Recursion) \textbf{[OL]}. https://www.quickperm.org/
\bibitem{bib_mersenne_randgen} Matsumoto, M.; Nishimura, T. Dynamic creation of pseudorandom number generators \textbf{[J]}. \textit{Monte Carlo and Quasi-Monte Carlo Methods 2000,} 1998: 56–69.
\bibitem{bib_msvc_threads} C11 Threads in Visual Studio 2022 version 17.8 Preview 2 \textbf{[OL]}. https://devblogs.microsoft.com/cppblog/c11-threads-in-visual-studio-2022-version-17-8-preview-2/

\end{thebibliography}

\appendix

\section{附录: 排序算法的 C 语言实现}

在本附录中, 我们列出本文所研究的排序算法的完整 C 语言实现. 本附录中的所有函数都具有如下所示的函数原型:

\lstinputlisting[language=C]{../algo/common.h}

\subsection{冒泡排序}

\lstinputlisting[language=C]{../algo/algo_bubblesort.c}

\subsection{直接插入排序}

\lstinputlisting[language=C]{../algo/algo_insertion_sort.c}

\subsection{直接选择排序}

\lstinputlisting[language=C]{../algo/algo_selection_sort.c}

\subsection{归并排序}

\lstinputlisting[language=C]{../algo/algo_merge_sort.c}

\subsection{以最末位元素为主元的快速排序}

\lstinputlisting[language=C]{../algo/algo_quicksort.c}

\subsection{由 \texttt{stdlib.h} 提供的 \texttt{qsort}}

\lstinputlisting[language=C]{../algo/algo_stdlib_qsort.c}

\subsection{基于 QuickPerm 的确定性 Bogosort}

\lstinputlisting[language=C]{../algo/algo_bogosort.c}

\subsection{扩展基数排序}

\lstinputlisting[language=C]{../algo/algo_eradix_sort.c}

\section{附录: 测试驱动程序}

在本附录中, 我们给出得到实验结果时所使用的测试驱动程序. 该程序向 \texttt{stdout} 输出 CSV 格式的机器可读结果信息, 同时向 \texttt{stderr} 输出人可读的进度与结果信息.

\lstinputlisting[language=C]{../test/main.c}

\section{附录: 源代码可见性}

本报告中使用的 C 语言源代码可于 \url{https://github.com/CSharperMantle/sort_algos_perf} 处获得, 并在 BSD 3-Clause 许可证的条件与条款下使用. 关于授权协议的具体信息请见上述 URL.

\end{document}
